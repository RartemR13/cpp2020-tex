\section{Лекция 1}

\subsection*{О лекторе}
Лектора зовут Мещерин Илья Семирович.

VK: vk.com/mesyarik

TG: @mesyarik

\subsection{Общие слова о языке и исторические заметки}
Почему изучается именно этот язык программирования и что он даст?
На сайте www.tiobe.com в котором собрана информация о популярности языков программирования C++ занимает достаточно высокое место.

Существуют языки \textbf{общего} и \textbf{специального} назначений,
например на C++, C\#, Python, Java можно писать почти все что угодно, а на JavaScript~--- нет.

По этому курсу будет довольно много лекций, потому что C++ довольно большой язык.
Вместе с изучением C++ будет затронут и его прородитель~--- C.

Почти всегда то, что написано на C++ будет работать и на C.
Также, изучая C++ будет намного проще изучать Java и C\#, потому что эти языки синтаксически похожи.

Язык C++~--- очень сложен, освоив его изучать остальное будет проще.
Сложность освоения C++ связана со многими вещами, как минимум он довольно низкоуровневый.
Придется следить за памятью, есть возможность общаться с ОС на прямую и т.д.

В Java и C\# мы защищены от низкоуровневых инструкций, что делают их проще и безопаснее.
Однако в этом есть и плюсы и минусы. Из-за незащищенности и низкоуровневости код C++ может работать в разы быстрее своих аналогов.
Исходя из этого C++ используется там, где нужен высокопроизводительный код.

Примеров известных сервисов, которые написаны с использованием кода на C++ довольно много.

Яндекс: Поиск, Карты, Такси, Почта, Диск.

Google: Search, Youtube.

Телеграмм.

Также на C++ написано множество игр, например WoW.

Не стоит и забывать про ОС, даже Windows написан с использованием C++.

C++ используется и в науке.


С++ создал Бьерн Страуструп (Bjarne Stroustrup).
В курсе его будем называть \textbf{создателем}.

У C++ много версий. Мы будем пользоваться стандартом C++ 17.

\begin{itemize}
    \item C++98
    \item C++03

    ------------

    \item C++11 (С++0x)
    \item C++14 (C++1y)
    \item C++17 (C++1z)
\end{itemize}

Между версиями языка не зря проведена черта. Они очень сильно отличаются.
У каждой версии есть \textbf{стандарт}. Тонны страниц текста с мелким шрифтом, исчерпывающе описывающим версию языка.
Из-за его объема и формальности не рекомендуется читать и вникать в него полностью, а лишь обращаться к нему и читать по диагнали в неоднозначных моментах.

Рекомендуется пользоваться сайтом www.cppreference.com.
Если требуется что-то узнать о языке, то первым делом лучше лезть туда.

Не стоит брезговать сайтом www.stackoverflow.com, на котором почти всегда будут ответы на возникающие вопросы.

Рекомендованная литература, автора Скота Мейерса:
\begin{itemize}
    \item Эффективное использование C++.
    \item Наиболее эффективное использование C++.
    \item Эффективный и современный C++.
\end{itemize}

\subsection{Знакомство с компилятором и первая программа}
Добро пожаловать в терминал Linux!
В курсе рекомендуется пользоваться именно этой операционной системой.

Напишем первую программму.

\begin{minted}{cpp}
#include <iostream>

int main() {
    int x;
    std::cin >> x;
    std::cout << x + 5 << '\n';
}
\end{minted}


В программе на C++ обязательно должна быть такая функция как \textbf{main}.
Здесь мы пишем программу от которой ожидаем простые действия, а именно ввести число, увеличить его на пять и вывести.

Чтобы делать ввод-вывод необходимо подключить \textbf{заголовочный файл}~--- \underline{iostream}.

Написав строку 
\begin{minted}{cpp}
#include <iostream>
\end{minted}
мы подключаем заголовочный файл и получаем возможность работать со стандартными потоками ввода-вывода.

Для того чтобы ввести переменную, необходимо ее \textbf{объявить}. 
\begin{minted}{cpp}
    int x;
\end{minted}

Чтобы ввести переменную из стандартного потока ввода пишем
\begin{minted}{cpp}
    std::cin >> x;
\end{minted}

а чтобы вывести число в стандартный поток вывода из этой переменной увеличенное на 5 пишем
\begin{minted}{cpp}
    std::cout << x + 5 << '\n';
\end{minted}

Здесь мы также добавляем к выводу \textbf{символ перевода строки}~--- \underline{'$\backslash$n'}.

Чтобы перевести строку можно написать и 
\begin{minted}{cpp}
    std::cout << x + 5 << std::endl;
\end{minted}

однако есть разница. Во втором случае мы также отчищаем буффер вывода.

Файлы исходного кода обычно выглядят как \underline{*.cpp}.

Перейдем к компиляции. Для того чтобы получить исполняемую программу необходимо 
перевести исходный код в машинный, этот процесс называется \textbf{компиляцией}, а
наличие этого процесса отличает компилируемые языки программирования от интерпретируемых.

Для того чтобы скомпилировать исходный код необходимо вызвать специальную программу~--- \textbf{компилятор}.
У C++ есть несколько компиляторов. Наиболее известным из них является \underline{gcc}~--- Gnu Compiler Collection.
Для того чтобы его вызвать именно для языка С++ необходимо писать g++.

Также есть такие компиляторы, как clang++ и msvc.
Компилятор msvc разрабатывается компанией Microsoft и не рекомендован к использованию,
поскольку многие вещи от туда не всегда соответсвуют стандарту.

Для отладки программы можно пользоваться программой дебагером, например \underline{gdb}.